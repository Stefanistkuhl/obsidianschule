\documentclass[a4paper]{article}
\usepackage{graphicx}
\usepackage{tabularx}
\usepackage{fancyhdr}
\usepackage{geometry}
\geometry{
	a4paper,
	total={170mm,257mm},
	left=20mm,
	top=20mm,
	bottom=39mm,
}

\setlength{\headheight}{82.70538pt}

\fancypagestyle{oida}{
	\fancyhf{}
	\fancyhead[L]{\fontsize{7.5}{7.5}htl donaustadt\\ Donaustadtstraße 45\\
		1220 Wien\\~\\ Abteilung: Informationstechnologie\\ 
	Schwerpunkt: Netzwerktechnik}
	\fancyhead[R]{\includegraphics[scale=0.45]{logo.png}}

	\fancyfoot[L]{\today}
	\fancyfoot[C]{\jobname}
	\fancyfoot[R]{Seite: \thepage}
}

\begin{document}
\pagestyle{oida}
\section*{Thema}
\par\noindent\rule{\textwidth}{0.4pt}

Laborprotokoll

\begin{figure}[h]
	\includegraphics[scale=0.6]{meme.jpeg}
	\caption{memes klauen ist nicht ethisch}
\end{figure}

\vspace*{\fill}
Unterrichtsgegenstand:	NWT1|ZIVK

Jahrgang:	2BHIT

Name:	Stefan Fürst

Betreuer: 	ZIVK

Übungsdaten:	Datum

Abgabedatum:	Datum


\newpage
\tableofcontents

\newpage

\section{Aufgabenstellung}
Statisches Routing in einem großen Netzwerk einzeichnen, Routing Tabellen machen und durch die GUI eine Kommunikation ermöglichen.
\section{Zusammenfassung}
Tonnenweise Pfeile zeichnen und Tabellen machen und Packet Tracer UX genießen.
\newpage

\section{Vollständige Netzwerktopologie der gesamten Übung}

\begin{figure}[h]
	\includegraphics[scale=0.5]{topologie.png}
	\caption{Netzwerktopologie der Übung}
\end{figure}

\newpage

\section{Übungsdurchführung}

\subsection{Routing Tabellen}
\subsubsection{Pc 5 zu Server 1}
Pc 5 Routing Tabelle
\begin{center}
	\begin{tabularx}{0.69\textwidth} { 
			| >{\raggedright\arraybackslash}X 
			| >{\centering\arraybackslash}X| 
		}
		\hline
		Ziel Netz & Nexthop\\
		\hline
		0.0.0.0/0 & 10.0.25.254/24\\
		\hline
	\end{tabularx}
\end{center}

Router 9 Routing Tabelle
\begin{center}
	\begin{tabularx}{0.69\textwidth} { 
			| >{\raggedright\arraybackslash}X 
			| >{\centering\arraybackslash}X| 
		}
		\hline
		Ziel Netz & Nexthop\\
		\hline
		10.0.26.0/24 & 10.0.25.253/24\\
		\hline
	\end{tabularx}
\end{center}

\subsubsection{Rückweg}
Server 1 Routing Tabelle
\begin{center}
	\begin{tabularx}{0.69\textwidth} { 
			| >{\raggedright\arraybackslash}X 
			| >{\centering\arraybackslash}X| 
		}
		\hline
		Ziel Netz & Nexthop\\
		\hline
		0.0.0.0/0 & 10.0.26.254/24\\
		\hline
	\end{tabularx}
\end{center}
Router 12 Routing Tabelle
\begin{center}
	\begin{tabularx}{0.69\textwidth} { 
			| >{\raggedright\arraybackslash}X 
			| >{\centering\arraybackslash}X| 
		}
		\hline
		Ziel Netz & Nexthop\\
		\hline
		10.0.25.0/24 & 10.0.24.254/24\\
		\hline
	\end{tabularx}
\end{center}
\subsubsection{Pc 0 zu Server 0}
Pc 0 Routing Tabelle
\begin{center}
	\begin{tabularx}{0.69\textwidth} { 
			| >{\raggedright\arraybackslash}X 
			| >{\centering\arraybackslash}X| 
		}
		\hline
		Ziel Netz & Nexthop\\
		\hline
		0.0.0.0/0 & 10.0.1.254/24\\
		\hline
	\end{tabularx}
\end{center}
Router 0 Routing Tabelle
\begin{center}
	\begin{tabularx}{0.69\textwidth} { 
			| >{\raggedright\arraybackslash}X 
			| >{\centering\arraybackslash}X| 
		}
		\hline
		Ziel Netz & Nexthop\\
		\hline
		10.0.4.0/24 & 10.0.2.253/24\\
		\hline
	\end{tabularx}
\end{center}
Router 1 Routing Tabelle
\begin{center}
	\begin{tabularx}{0.69\textwidth} { 
			| >{\raggedright\arraybackslash}X 
			| >{\centering\arraybackslash}X| 
		}
		\hline
		Ziel Netz & Nexthop\\
		\hline
		10.0.4.0/24 & 10.0.3.253/24\\
		\hline
	\end{tabularx}
\end{center}
\subsubsection{Rückweg}
Server 0 Routing Tabelle
\begin{center}
	\begin{tabularx}{0.69\textwidth} { 
			| >{\raggedright\arraybackslash}X 
			| >{\centering\arraybackslash}X| 
		}
		\hline
		Ziel Netz & Nexthop\\
		\hline
		0.0.0.0/0 & 10.0.4.254/24\\
		\hline
	\end{tabularx}
\end{center}
Router 2 Routing Tabelle
\begin{center}
	\begin{tabularx}{0.69\textwidth} { 
			| >{\raggedright\arraybackslash}X 
			| >{\centering\arraybackslash}X| 
		}
		\hline
		Ziel Netz & Nexthop\\
		\hline
		10.0.1.0/24 & 10.0.3.254/24\\
		\hline
	\end{tabularx}
\end{center}
Router 1 Routing Tabelle
\begin{center}
	\begin{tabularx}{0.69\textwidth} { 
			| >{\raggedright\arraybackslash}X 
			| >{\centering\arraybackslash}X| 
		}
		\hline
		Ziel Netz & Nexthop\\
		\hline
		10.0.1.0/24 & 10.0.2.254/24\\
		\hline
	\end{tabularx}
\end{center}
\newpage
\subsection{Einstellen der Statischenn Routen in Packet Tracer mit der Gui.}
Auf den Router doppelclicken, dann im config Tab unter Routing bei Static die richtigen IP Adressen eintragen.
\begin{figure}[h]
	\centering
	\includegraphics[scale=0.5]{ptgui.png}
	\caption{Statische Route festlegen}
\end{figure}
\newpage
\subsection{Routen von jedem Pc zu jedem Servernetzwerk}
\begin{figure}[h]
	\centering
	\includegraphics[scale=0.5]{1.jpg}
	\caption{1/12}
\end{figure}
\begin{figure}[h]
	\centering
	\includegraphics[scale=0.5]{2.jpg}
	\caption{2/12}
\end{figure}
\begin{figure}[h]
	\centering
	\includegraphics[scale=0.5]{3.jpg}
	\caption{3/12}
\end{figure}
\begin{figure}
	\centering
	\includegraphics[scale=0.5]{4.jpg}
	\caption{4/12}
\end{figure}
\begin{figure}
	\centering
	\includegraphics[scale=0.5]{5.jpg}
	\caption{5/12}
\end{figure}
\begin{figure}
	\centering
	\includegraphics[scale=0.5]{6.jpg}
	\caption{6/12}
\end{figure}
\begin{figure}
	\centering
	\includegraphics[scale=0.5]{7.jpg}
	\caption{7/12}
\end{figure}
\begin{figure}
	\centering
	\includegraphics[scale=0.5]{8.jpg}
	\caption{8/12}
\end{figure}
\begin{figure}
	\centering
	\includegraphics[scale=0.5]{9.jpg}
	\caption{9/12}
\end{figure}
\begin{figure}
	\centering
	\includegraphics[scale=0.5]{10.jpg}
	\caption{10/12}
\end{figure}
\begin{figure}
	\centering
	\includegraphics[scale=0.5]{11.jpg}
	\caption{11/12}
\end{figure}
\begin{figure}
	\centering
	\includegraphics[scale=0.5]{12.jpg}
	\caption{12/12}
\end{figure}

\newpage

\section{Abbildungsverzeichnis}

\listoffigures

\end{document}
