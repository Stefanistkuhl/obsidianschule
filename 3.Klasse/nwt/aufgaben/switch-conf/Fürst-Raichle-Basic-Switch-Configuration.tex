\documentclass[a4paper]{article}
\usepackage{graphicx}
\usepackage{xcolor}
\usepackage{url}
\usepackage{outlines}
\usepackage{listings}
\usepackage{fontspec}
% \setmainfont{Calibri}
\lstset{basicstyle=\ttfamily,
	showstringspaces=false,
	commentstyle=\color{blue},
	keywordstyle=\color{pink}
}
\lstset{emph={
	nc,tcp,udp,http,},emphstyle=\color{purple}
}
\usepackage{fancyhdr}
\usepackage{geometry}
\geometry{
	a4paper,
	total={170mm,257mm},
	left=20mm,
	top=20mm,
	bottom=39mm,
}

\setlength{\headheight}{82.70538pt}

\fancypagestyle{oida}{
	\fancyhf{}
	\fancyhead[L]{\fontsize{7.5}{7.5}htl donaustadt\\ Donaustadtstraße 45\\
		1220 Wien\\~\\ Abteilung: Informationstechnologie\\ 
	Schwerpunkt: Netzwerktechnik}
	\fancyhead[R]{\includegraphics[scale=0.45]{images/logo.png}}

	\fancyfoot[L]{\today}
	\fancyfoot[C]{\jobname}
	\fancyfoot[R]{Seite: \thepage}
}

\begin{document}
\bibliographystyle{plain}
\pagestyle{oida}
\section*{Thema}
\par\noindent\rule{\textwidth}{0.4pt}

Laborprotokoll
Template

\begin{figure}[h]
	\includegraphics[scale=0.3]{images/mika.jpeg}
	\caption{Wunderbares Gruppenlogo}
\end{figure}

\vspace*{\fill}
Unterrichtsgegenstand:	NWT|ZIVK

Jahrgang:	3AHITN

Name:	Stefan Fürst, Marcel Raichle

Gruppenname/Nummer: Dumm und Dümmer/7

Betreuer: 	ZIVK

Übungsdaten:

Abgabedatum:	


\newpage
\tableofcontents

\newpage

\section{Aufgabenstellung}



\section{Zusammenfassung}

\newpage

\section{Vollständige Netzwerktopologie der gesamten Übung}

\newpage

\section{Übungsdurchführung}
%überschriften auf deutsch übersetzen
\subsection{Cable the Nework and Verify the Default Switch Configuration}
2b
	24 fe interfaces
		2 gbe ifs
		vty range
			5 15
2c 
	show startup-config
	startup-config is not present
	keine start conf wurde im NVRAM gespeichert
2d 
	keine ip in vlan1 per default
	mac addr
	0018.18bc.59c0
	its up
2e	
	show ip interface vlan1
	Internet protocal processing disabled
2f
	show ip interface vlan1
	Vlan1 is up, line protocol is up
	Internet protocol processing disabled
2g
	show version
	version:12.12
	image filename:
	show flash
	c3560-ipservicesk9-mz.122-55.SE6.bin
	mac addr:00:18:18:BC:59:80
2h
	show interface f0/6
	interface is up
	no shutdown/host connecten
	port mac 0018.18bc.5988
	Full-duplex, 100Mb/s,
2i
	Vlan1
	all of the ports bc every port is per default in vlan1
	its active
	Ethernet
2j 
	either show/dir flash
	show flash
		c3560-ipservicesk9-mz.122-55.SE6.bin
	dir flash

\subsection{Configure Basic Network Device Settings}
\paragraph {Configure basic switch settings}

a
no ip domain-lookup
hostname S1
service password-encryption
enable secret class
banner motd #
Unauthorized access is strictly prohibited. #

b
conf t
vlan 99 
exit
interface vlan99
ip address 192.168.1.2 255.255.255.0
ipv6 address 2001:db8:acad::2/64
ipv6 address fe80::2 link-local
no shutdown

c
interface range f0/1 – 24,g0/1 - 2
switchport access vlan 99

d
ss auf dc

e
ip default-gateway 192.168.1.1

f
line con 0 
logging synchronous
password cisco
login

g
ohne login command, wird nicht nach einem passwort gefragt


\paragraph {configure the pc ip addr}
step2:
ss in dc für beides
\subsection{Verify and Test Network Connectivity}
a
show run
b
show interface vlan99
bandwith: 1000000 Kbit
vlan99 state: up
line state: down



\subsection{Manage the MAC Address Tabel}

step1: 
D8-43-AE-86-06-88
step2:
show mac address-table
wv dynamische mac addr
1
wv gesamt:
21
is dasselbe:
ja
step3:
a
13

\newpage

\section{Vollständige Konfigurationsdateien}

\newpage

\section{Quellen}
\bibliography{quellen}
\newpage
\section{Abbildungsverzeichnis}

\listoffigures

\end{document}
