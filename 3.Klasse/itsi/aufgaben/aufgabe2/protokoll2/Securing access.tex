\documentclass[a4paper]{article}
\usepackage{graphicx}
\usepackage{xcolor}
\usepackage{url}
\usepackage{outlines}
\usepackage{listings}
\usepackage{fontspec}
\lstset{basicstyle=\ttfamily,
	showstringspaces=false,
	commentstyle=\color{blue},
	keywordstyle=\color{pink}
}
\lstset{emph={
	EXPOSE,RUN,FROM,CMD,nc,tcp,udp,http,docker},emphstyle=\color{purple}
}
\newcommand{\abc}{\hfill \break}
\usepackage{fancyhdr}
\usepackage{geometry}
\geometry{
	a4paper,
	total={170mm,257mm},
	left=20mm,
	top=20mm,
	bottom=39mm,
}

\setlength{\headheight}{82.70538pt}

\fancypagestyle{oida}{
	\fancyhf{}
	\fancyhead[L]{\fontsize{7.5}{7.5}htl donaustadt\\ Donaustadtstraße 45\\
		1220 Wien\\~\\ Abteilung: Informationstechnologie\\ 
	Schwerpunkt: Netzwerktechnik}
	\fancyhead[R]{\includegraphics[scale=0.45]{images/logo.png}}

	\fancyfoot[L]{\today}
	\fancyfoot[C]{\jobname}
	\fancyfoot[R]{Seite: \thepage}
}

\begin{document}
\bibliographystyle{plain}
\pagestyle{oida}
\section*{Thema}
\par\noindent\rule{\textwidth}{0.4pt}

Laboratory protocol
GNU/Linux - Securing access

\begin{figure}[h]
	\includegraphics[scale=0.3]{images/mika.jpeg}
	\caption{Grouplogo}
\end{figure}

\vspace*{\fill}
Subject:	ITSI|ZIVK

Class:	3AHITN

Name:	Stefan Fürst, Marcel Raichle

Gruppenname/Nummer: Dumm und Dümmer/7

Supervisor: 	ZIVK

Exercise dates:	

Submission date:


\newpage
\tableofcontents

\newpage

\section{Task definition}



\section{Summary}


\newpage

\section{Exercise Execution}

%give better title
\subsection{Privileged rights}
\subsubsection{Explanation of the sudo command}
The \texttt{sudo} command or \textbf{S}uper\textbf{U}ser \textbf{DO} temporarily elevates privileges and runs the set command as root, which can be seen by running the \texttt{sudo id} command.\cite{sudo}

\begin{figure}[h]
	\centering
	\includegraphics[scale=0.4]{images/sudoid.png}
	\caption{sudo id}
\end{figure} \abc
As seen in the figure, when the \texttt{id} command is used with \texttt{sudo}, the id displayed is 0, which is the user id of the root user, and without sudo it displays the normal user id of the user who executed the command.

\subsubsection{Granting and restricting users' sudo access}
To grant someone permission to run any command with \texttt{sudo}, the \texttt{usermod -aG sudo username} command is used, which appends the given to the sudo group, giving them permission to run any command with sudo. \abc
In order to restrict the commands that can be elevated by a user or to configure other settings related to this, it is necessary to edit the configuration file, which is located at \texttt{/etc/sudoers}.\abc
There are several ways to edit it. The \texttt{visudo} command uses the editor set in the \texttt{\$EDITOR} environment variable and opens the sudoers file with it, and when you exit the editor and save it, it also checks for errors before applying the changes.
The sudoers file can also be directly edited using \texttt{echo} in the dockerfile.
\begin{lstlisting}[language=bash]
#only allowing ram-alois to edit the ssh configuration file
RUN echo "ram-alois ALL=(root) /bin/nano /etc/ssh/sshd_config" >> /etc/sudoers
#only allowing ram-berta to add users
RUN echo "ram-berta ALL=(root) /sbin/useradd" >> /etc/sudoers
#only allowing to ram-ram to view and read add files
RUN echo "ram-ram ALL=(root) /bin/ls" >> /etc/sudoers
RUN echo "ram-ram ALL=(root) /bin/cat" >> /etc/sudoers
\end{lstlisting}
I chose nano over vim for editing the ssh config file, as running vim as sudo effectively gives the user full sudo access, as it is possible to open a terminal in it and escape the normal editor mode in numerous ways, so its just easier to give the user nano.\abc
insert screenshots of the thing
\subsubsection{Setting up a password policy}

\begin{lstlisting}[language=bash]
RUN sed -i 's/#PermitRootLogin prohibit-password/PermitRootLogin yes/' /etc/ssh/sshd_config
RUN sed -i '/retry=3/ s/$/ucredit=-1 dcredit=-1 ocredit=-1/' /etc/pam.d/common-password
RUN sed -i '/ocredit=-1/ a password\trequisite\t\t\tpam_pwhistory.so remember=5 use_authtok' /etc/pam.d/common-password
RUN sed -i '/yescrypt/ s/$/ minlen=10/' /etc/pam.d/common-password
\end{lstlisting}

\newpage
\section{References}
\bibliography{quellen}
\newpage
\section{List of figures}

\listoffigures

\end{document}
