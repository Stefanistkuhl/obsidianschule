\documentclass[a4paper]{article}
\usepackage{graphicx}
\usepackage{xcolor}
\usepackage{url}
\usepackage{outlines}
\usepackage{listings}
\usepackage{fontspec}
% \setmainfont{Calibri}
\lstset{basicstyle=\ttfamily,
	showstringspaces=false,
	commentstyle=\color{gray},
	keywordstyle=\color{pink}
}
\lstset{emph={
	nc,tcp,udp,http,openssl,sha256sum,},emphstyle=\color{purple}
}
\usepackage{fancyhdr}
\usepackage{geometry}
\geometry{
	a4paper,
	total={170mm,257mm},
	left=20mm,
	top=20mm,
	bottom=39mm,
}

\setlength{\headheight}{82.70538pt}

\fancypagestyle{oida}{
	\fancyhf{}
	\fancyhead[L]{\fontsize{7.5}{7.5}htl donaustadt\\ Donaustadtstraße 45\\
		1220 Wien\\~\\ Abteilung: Informationstechnologie\\ 
	Schwerpunkt: Netzwerktechnik}
	\fancyhead[R]{\includegraphics[scale=0.45]{images/logo.png}}

	\fancyfoot[L]{\today}
	\fancyfoot[C]{\jobname}
	\fancyfoot[R]{Seite: \thepage}
}

\begin{document}
\bibliographystyle{plain}
\pagestyle{oida}
\section*{Kryptographie}
\par\noindent\rule{\textwidth}{0.4pt}

Laborprotokoll
Übung 3

\begin{figure}[h]
	\includegraphics[scale=0.3]{images/mika.jpeg}
	\caption{Wunderbares Gruppenbild}
\end{figure}

\vspace*{\fill}
Unterrichtsgegenstand:	ITSI|ZIVK

Jahrgang:	3AHITN

Name:	Stefan Fürst, Marcel Raichle

Gruppenname/Nummer: Dumm und Dümmer/7

Betreuer: 	ZIVK

Übungsdaten:	24.5.2024, 31.5.2024,7.6.2024

Abgabedatum:	7.6.2024


\newpage
\tableofcontents

\newpage

\section{Aufgabenstellung}



\section{Zusammenfassung}


\newpage

\section{Übungsdurchführung}
todo: zu jedem command erklären was er macht und die genutzen optionen
\subsection{Symmetrisch Verschlüsseln}
\begin{lstlisting}[language=bash]
#verschlüsseln
openssl aes256 -in Raichle.txt -out Raichle.encrypted
#entschlüsseln
openssl aes256 -d -in Raichle.encrypted -out Raichle.txt
\end{lstlisting}

\subsection{Asymmetrisch Verschlüsseln}
\begin{lstlisting}[language=bash]
#public key
openssl genpkey -algorithm RSA -pkeyopt rsa_keygen_bits:4096 -out private-key.pem
#private key
openssl pkey -in private-key.pem -out public-key.pem -pubout
#datei verschlüsseln
openssl rsautl -encrypt -inkey zivk.pem -pubin -in Raichle-Fuerst-RSA.txt -out Raichle-Fuerst-RSA.txt.zivk.enc
\end{lstlisting}

\subsection{Integrität prüfen}
\begin{lstlisting}[language=bash]
#benötigter command
sha256sum <dateiname>
\end{lstlisting}

\begin{figure}[h]
	\includegraphics[scale=0.3]{images/hashes.png}
	\caption{Hashes}
\end{figure}

% der 3te hat keinen hash


\newpage
\section{Quellen}
\bibliography{quellen}
\newpage
\section{Abbildungsverzeichnis}

\listoffigures

\end{document}
