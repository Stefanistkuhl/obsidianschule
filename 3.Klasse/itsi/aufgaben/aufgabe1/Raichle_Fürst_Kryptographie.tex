\documentclass[a4paper]{article}
\usepackage{graphicx}
\usepackage{xcolor}
\usepackage{url}
\usepackage{outlines}
\usepackage{listings}
\usepackage{fontspec}
% \setmainfont{Calibri}
\lstset{basicstyle=\ttfamily,
	showstringspaces=false,
	commentstyle=\color{blue},
	keywordstyle=\color{pink}
}
\lstset{emph={
	nc,tcp,udp,http,openssl,},emphstyle=\color{purple}
}
\usepackage{fancyhdr}
\usepackage{geometry}
\geometry{
	a4paper,
	total={170mm,257mm},
	left=20mm,
	top=20mm,
	bottom=39mm,
}

\setlength{\headheight}{82.70538pt}

\fancypagestyle{oida}{
	\fancyhf{}
	\fancyhead[L]{\fontsize{7.5}{7.5}htl donaustadt\\ Donaustadtstraße 45\\
		1220 Wien\\~\\ Abteilung: Informationstechnologie\\ 
	Schwerpunkt: Netzwerktechnik}
	\fancyhead[R]{\includegraphics[scale=0.45]{images/logo.png}}

	\fancyfoot[L]{\today}
	\fancyfoot[C]{\jobname}
	\fancyfoot[R]{Seite: \thepage}
}

\begin{document}
\bibliographystyle{plain}
\pagestyle{oida}
\section*{Kryptographie}
\par\noindent\rule{\textwidth}{0.4pt}

Laborprotokoll
Übung 3

\begin{figure}[h]
	\includegraphics[scale=0.3]{images/mika.jpeg}
	\caption{Wunderbares Gruppenbild}
\end{figure}

\vspace*{\fill}
Unterrichtsgegenstand:	ITSI|ZIVK

Jahrgang:	3AHITN

Name:	Stefan Fürst, Marcel Raichle

Gruppenname/Nummer: Dumm und Dümmer/7

Betreuer: 	ZIVK

Übungsdaten:	24.5.2024, 31.5.2024,7.6.2024

Abgabedatum:	7.6.2024


\newpage
\tableofcontents

\newpage

\section{Aufgabenstellung}



\section{Zusammenfassung}


\newpage

\section{Übungsdurchführung}

\subsection{Symmetrisch Verschlüsseln Übung 1}
zum verschlüsseln:
\begin{lstlisting}
openssl aes256 -in Raichle.txt -out Raichle.encrypted
\end{lstlisting}
zum entschlüsseln

\subsection{Asymmetrisch Verschlüsseln Übung 1}
\texttt{openssl aes256 -d -in Raichle.encrypted -out Raichle.txt}
Private key
\begin{lstlisting}
openssl genpkey -algorithm RSA -pkeyopt rsa_keygen_bits:4096 -out private-key.pem
openssl pkey -in private-key.pem -out public-key.pem -pubout
\end{lstlisting}
Pub key

datei verschlüsseln
openssl rsautl -encrypt -inkey zivk.pem -pubin -in Raichle-Fuerst-RSA.txt -out Raichle-Fuerst-RSA.txt.zivk.enc
\subsection{Integrität prüfen}
command:
\texttt{sha256sum <dateiname>}
Hash erste datei
e605bd1f525b133340d704f0e899d977f37dea63c14b243a346f1b524499bcf5
4te aus der liste
2.
591ad652f7332fdca28e4ecc520ad7b71852cdaa7ab3efbaeeb6042a815c812d
1te aus der liste
3.
e05c11789a98a495d7283a499b1ccc31c368d3c191a4fb5b7074161c816da2e3
2te aus der liste
4.
a20fb601802f7b87b2063964e0d2f7e15b2448bc6ee64dd7a9099991723a2666
5te aus der liste

der 3te hat keinen hash



\newpage
\section{Quellen}
\bibliography{quellen}
\newpage
\section{Abbildungsverzeichnis}

\listoffigures

\end{document}
